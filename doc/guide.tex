\documentclass[12pt]{article}
\usepackage[english]{babel}

\usepackage{graphicx}
\usepackage[showframe=false]{geometry}
\usepackage{amsfonts}

\begin{document}

\begin{center}
\textsc{\LARGE Czech Technical University in Prague}\\[0.5cm]
\textsc{\Large Faculty of electrical engineering}\\[0.5cm]
\textsc{\Large Pager - User Guide}\\[0.5cm]
\textsc{\large A0B36APO}\\[0.5cm]

\begin{minipage}{0.4\textwidth}
\begin{flushleft} \large
\emph{Author:}\\
Denys \textsc{Rozumnyi} 
\end{flushleft}
\end{minipage}
~
\begin{minipage}{0.4\textwidth}
\begin{flushright} \large
\emph{Author:} \\
Kamil \textsc{Svec}
\end{flushright}
\end{minipage}\\[0.5cm]


\begin{center}
{\large \today}
\end{center}
\end{center}

\section{Description}
This documentation provides user manual to PCIe card EVB/Altera DB4CGX15 software together with APO PagerCentral software. 

\section{Start}
First of all, you must run command \textbf{make}, which will compile the source code. Then you need to know server address or you can run your own using APO PagerCentral. Our program has two inputs:
\begin{enumerate}
\item Pager ID (default value $10$)
\item Server address (default value $127.0.0.1$)
\end{enumerate} 

\section{Usage}
When started the pager is in the stand-by state. In this state every 5 seconds it checks if there are any incoming messages on the server. If there is a message, the pager beeps and displays the message. The pager also allows to send a message to other pager in the system. While in stand-by mode, a user can press any key to enter the messaging mode. First he/she has to enter other pager's ID and then enter the message. So far the message consists only of numbers. Also you can remove numbers. 


\end{document}
