\documentclass[12pt]{article}
\usepackage[english]{babel}

\usepackage{graphicx}
\usepackage[showframe=false]{geometry}
\usepackage{amsfonts}
\usepackage{enumitem}

\begin{document}

\begin{center}
\textsc{\LARGE Czech Technical University in Prague}\\[0.5cm]
\textsc{\Large Faculty of electrical engineering}\\[0.5cm]
\textsc{\Large Pager - Documentation}\\[0.5cm]
\textsc{\large A0B36APO}\\[0.5cm]

\begin{minipage}{0.4\textwidth}
\begin{flushleft} \large
\emph{Author:}\\
Denys \textsc{Rozumnyi} 
\end{flushleft}
\end{minipage}
~
\begin{minipage}{0.4\textwidth}
\begin{flushright} \large
\emph{Author:} \\
Kamil \textsc{Svec}
\end{flushright}
\end{minipage}\\[0.5cm]


\begin{center}
{\large \today}
\end{center}

\end{center}

\section{Introduction}
The code is distributed into several files. Here we will describe each of them and the role they play.

\section{pci.c}
It is a starting point of the program, e.g. it has \textbf{main} function which has the main loop of the program. It calls all functions and is alternating between default mode and message mode. Also it starts all threads: for message receiving and keyboard listener, which is waiting until the user writes \textbf{end} in standard input and will finish the application. 

\section{pci\_lcd.c}
It is used for initiation of LCD display, writing to it, for reading from keyboard and displaying on LEDs. Also there are some helpers, such as make a beep, translating a key, writing data to bus, reading data from bus.

\section{klient.c}
This file makes working with a server easier. It provides functions for connection to the server and implements a listener for the message receiver. 

\section{finder.h}
The file provides function findMe() which will initiate everything for searching, and returns found device afterwards. Function \textbf{bfs} implements breadth-first search in given directory.

\section{kbd\_hw.h}
There definitions of mmaped\_8bit\_bus\_kbd hardware are defined.

\section{chmod\_lcd.h}
It's a file with control codes for two or four line character LCD modules.

\end{document}
